\documentclass[a4paper, 12pt]{article}
\usepackage[utf8]{inputenc}
\usepackage[english, russian]{babel}

\usepackage{amssymb}
\usepackage{amsmath}
\usepackage{amsthm}
\usepackage{enumitem}
\setlist{nolistsep}
\usepackage{cmap}

\usepackage[
top=20mm,
left=20mm,
right=15mm,
bottom=20mm,
footskip=10mm
]{geometry}

\renewcommand{\baselinestretch}{1.0}

%\def \newcommand #1 #2 {\displaystyle \frac{#1}{#2}}


\newtheorem*{theorem*}{Теорема}
\newtheorem*{task*}{Задача}

\frenchspacing

\begin{document}
\title{Решение системы линейных уравнений блочным 
методом Холецкого}
\author{Бобровников~М.~Р.}
\date{}
\maketitle
\section{Постановка задачи}

\begin{task*}
    Найти решение системы линейных уравнений $A x=b$. Где $A $ --- симметричная 
    вещественнозначная матрицы размера $n \times n$, $b$ --- известный вектор размера
    $n$ и $x$ --- неизвестный вектор.
    Искать решение будем с помощью разложения Холецкого матрицы $A=R^T DR$.
    Далее найдем такое $y$, что $R^T y=b$. Затем найдем $x$ из условия $DR x=y$.
    Здесь $R$ --- верхнетреугольная матрица, $D$ --- диагональная матрица с $1$ или $-1$
    на диагонали.
\end{task*}

\begin{theorem*}
    Пусть матрица $A$ --- самосопряженная и все ее угловые миноры отличны
    от нуля. Тогда существует матрица $R=(r_{ij}) \in RT(n)$  с вещественными 
    положительными элементами на главной диагонали и диагональная матрица
    $D$ с вещественными равными по модулю единице дигональными элементами
    такие, что $A=R^T DR$.
\end{theorem*}

\begin{proof} [Решение задачи]
    
Применим точечный метод Холецкого для поиска матрицы $R$.
Элементы $d_{ii}$, $r_{ii}$ ,$r_{ij}$ могут быть вычиcлены по 
следующим формулам:
\begin{equation} \label{dot}
 d_{ii} = sgn(a_{ii}-\sum_{k=1}^{i-1}|r_{ki}|^2d_{kk}),\  i=1,...,n,
\end{equation}
$$
r_{ii} = \sqrt{|a_{ii}-\sum_{k=1}^{i-1}|r_{ki}|^2d_{kk}|},\  i=1,...,n,
$$
$$
r_{ij} = (a_{ij}-\sum_{k=1}^{i-1}r_{ki}d_{kk}r_{kj})/(r_{ii}d_{ii}),\  
i<j, \ i,j=1,...,n,
$$
\end{proof}

\section{Оценка сложности в алгоритме построения верхнетреугольной матрицы
в разложении Холецкого}

Из формул ~\eqref{dot} следует, что для вычисления элемента $d_{ii}$, $i=1,...,n$
требуется $i-1$ мультипликативных и столько же аддитивных операций. 
Следовательно, вычисление всех элементов матрицы $D$ требует
$\sum_{i=1}^n(i-1) = n(n-1)/2 = O(n^2)$ мультипликативных и столько 
аддитивных операций.

При фиксированном $i=1,...,n$ вычисление элементов $r_{ij}$ для всех 
$j=i+1,..n$ по формулам ~\eqref{dot} требует $1+\sum_{j=i+1}^n(i-1)=
(n-i)(i-1)+1$ мультипликативных и $\sum_{j=i+1}^n(i-1)=
(n-i)(i-1)$ аддитивных операций. Следовательно, вычисление всех
элементов матрицы $R$ требует $n$ операций извлечения корня,
$\sum_{i=1}^n((n-i)(i-1)+(i-1)+1)=n^3/6+O(n^2) \ (n \rightarrow \infty)$
мультипликативных и
$\sum_{i=1}^n((n-i)(i-1)+(i-1))=n^3/6+O(n^2) \ (n \rightarrow \infty)$
аддитивных операций.
Таким образом нахождение матрицы $R$ требует $n^3/3+O(n^2)\ (n \to \infty)$
арифметических операций.

\section{Описание блочного метода Холецкого}

Разобьем матрицу $A$ на блоки размера $m\times m$.
Из формулы $A = R^TDR$ ясно, что формулы для нахождения блоков 
матрицы $R$ имеют вид:
\begin{equation} \label{block}
 (R_{ii},\ D_{i}) = decomp(A_{ii} - \sum_{k=1}^{i-1}
R_{ki}^TD_kR_{ki}),\  i=1,...,n,
\end{equation}
$$
R_{ij} = D_i^{-1}(R_{ii}^T)^{-1}(A_{ij} - \sum_{k=1}^{i-1}
R_{ki}^TD_kR_{kj}),\  
i<j, \ i,j=1,...,n,
$$
Здесь $decomp(U)$ --- функция, которая вычисляет разложение
Холецкого симметричной матрицы $U$.
Из формул ~\eqref{block} видно, что для вычислений 
не нужно хранить дополнительные матрицы, кроме как одну --- для
сохранения результата нахождения $R_{ii}^{-1}$.
Результирующие матрицы $D_i,\ R_{ii}, \ ,R_{ij}$
можно хранить прямо на месте $A$ и $D$.




\subsection{Оценка сложности в алгоритме построения верхнетреугольной матрицы
в блочном разложении Холецкого}

Пусть $N=n/m$(т. е количество блоков в строке и столбце матрицы),
 а $m$ нацело делит $n$.
Сначала подсчитаем количество операций, требуемых для
 вычисления диагональных блоков $R_{ii}$.
Пусть сложность вычисления произведения матриц 
$Mult(n) = 2n^3-n^2$
\newline
Сложность разложения Холецкого
$Chol(n) = n^3/3$
\newline
Тогда сложность вычисления блока $R_{ii}$: 
$(i-1)Mult(m)+Chol(m)$
\newline

Сложность вычисления всех диагональных 
блоков $R_{ii}, i \in {1,...,N}$:
$$ S_1(n,m) = 
\sum_{i=1}^N ((i-1)Mult(m)+Chol(m))
= n^2m-n^2/2-2/3nm^2+nm/2$$
Найдем сложность вычисление всех блоков, стоящих над диагональю.
Для вычисления $R_{ki}^TD_kR_{kj}$ требуется $Mult(m)$ операций.
Для вычисления $A_{ij} - \sum_{k=1}^{i-1}R_{ki}^TD_kR_{kj}$ 
требуется $W(i,m)=\sum_{k=1}^{i-1}(Mult(m)+m^2)$ операций, т.к на 
каждом шаге нужно перемножать и вычитать матрицы.
Умножение на треугольную матрицу требует
$Y(n)=n^3$ операций.
Вычисление $R_{ij}: R(i, m) = W(i,m)+Y(m)=2im^3-m^3-m^2$.
Тогда $$S_1(n, m)=\sum_{i=1}^N\sum_{j=i+1}^N R(i,m) = 
\sum_{i=1}^N\sum_{j=i+1}^N (2im^3-m^3-m^2)=
1/6n(n-m)(2n-m-3)$$

Чтобы $(N-1)$ раз найти обратную к верхнетреугольной:
$S_3(n, m) = (N-1)m^3/3 = (n-m)m^2/3$

Итак, нахождение всех блоков $R_{ij}$ требует
$$S(n, m)= S_1 + S_2+S_3 = n^3/3-m^2n^2/6+2/3mn^2-n^2+m^3n/6-m^2n/3
+mn-m^3/3$$
\newline
$S(n,n)=n^3/3$
\newline
$S(n,1)=n^3/3+O(n^2),\ (n \to \infty )$

\subsection{Оптимизация хранения матриц для работы с кэшем процессора}
Так как матрица $A$ симметричная, то логично хранить не всю
матрицу, а только верхнюю ее часть над главной диагональю 
и саму диагональ. Матрицу можно расположить в памяти так, что
все блоки на диагонали будут иметь треугольный вид, а все 
остальные блоки будут либо квадратными размером
$m \times m$ или $l \times l$, либо прямоугольными размером
$m \times l$. Здесь $l$ --- размер последнего блока, равный
$n-(n/m)*m$, здесь $/$ обозначает деление нацело. Каждый блок 
линейно располагается в памяти. 
Таким образом все вычисления можно производить прямо на месте
блока $A_{ij}$ без копирования элементов блока. 


\section{Описание параллельного блочного метода Холецкого}

Считаем, что $p$ --- количество потоков и потоки никак не мешают друг другу.
В описанном выше линейном алгоритме мы находили блоки построчно.
В $i$-ой строке сначала получаем блок на диагонали,
затем находим обратную к нему обратную матрицу.
Все следующие на этой строке блоки получаются из уже вычисленных
блоков, находящихся только в предыдущих строках,
 и полученной обратной.
Таким образом, все блоки вне диагонали можно искать в 
любом порядке в контексте текущей строки.
\newline
Предлагается следующий параллельный алгоритм, в котором 
 каждый поток считает блоки только из 
 своего столбца.
Принадлежность столбца потоку определяется так:
 $i$-ый поток обрабатывает столбцы с номерами
 $i+m*p$, где $m \in \mathbb{Z}$
\begin{verbatim}
Для каждой строки i из {1,..,N}:
    Сохранение копии Temp = A_{ii}
    Синхронизация потоков
    Вычисление R_{ii}
    Вычисление R_{ii}^{-1}
    Если i = tid (mod p):
        Запись R_{ii} на место A_{ii}
    Для каждого столбца j из {i+1,..,N}:
        Если j = tid (mod p):
            Вычисление R_{ij}
            Запись R_{ij} на место A_{ij}
\end{verbatim}


\subsection{Описание параллельного решения системы $R^T y=b$}

Представляем, что $R^T$ на самом деле не транспонированная
и лежит в память как $R$. Но работаем с ней как с транспонированной.
Тогда получаем следующий алгоритм:
\begin{verbatim}
Для каждого i из {1,..,N}:
    Sum = 0
    Для каждой строки k из {1,.., i - 1}:
        Sum += R_{ki} * y[k]
    y[i] = (b[i] - Sum) / R_{ii}
\end{verbatim}

Сделаем его параллельным. Пусть имеется p потоков.          
Каждый поток подсчитывает свой $y[i]$. Правило такое: 
если $i \% p == tid$, то поток с номером $tid$ вычисляет
$y[i]$. Также 

\begin{verbatim}
void
compute_y_thread (mat a, vec y, vec b, int n, int m,
                    int tid, int p, 
                    pthread_cond_t *cond,
                    pthread_mutex_t *mut,
                    int *elems_done)
{
  for (int i = 0 + tid; i < n; i += p)
    {
      double sum = 0;
      int k;
      for (k = 0; k <= i - p; k++)
        {
          sum += a[n * k + i] * y[k];
        }
      pthread_mutex_lock (mut);
      while (*elems_done < i)
        {
          pthread_cond_wait (cond, mut);
        }
      pthread_mutex_unlock (mut);
      for (; k < i; k++)
        {
          sum += a[n * k + i] * y[k];
        }
      y[i] = (b[i] - total_sum) / a[i * n + i];
      pthread_mutex_lock (mut);
      (*elems_done)++;
      pthread_cond_broadcast (cond);
      pthread_mutex_unlock (mut);
    }
}
\end{verbatim}
Используется $n$ точек синхронизации.

Итак, был получен параллельный алгоритм. Причем каждый
поток использует только свою память, а синхронизация осуществляется
за счет использования условной переменной, где условие 
это количество уже полученных элементов $y[i]$.

Сложность обычного алгоритма составляет
$$\sum_{i=1}^n\sum_{k=1}^i2 = n*(n+1)=n^2+n$$

А многопоточного, в силу того, что никакой достаточно 
большой участок кода
не выполняется с заблокированным мьютексом
$$(n^2+n)/p$$

Теперь получим алгоритм поиска вектора $x$.
\begin{verbatim}
void
compute_x_thread (mat a, vec x, vec y, vec d, int n, int m,
                  int tid, int p,
                  int *sh_thread_sum)
{
  for (int i = n - 1; i >= 0; i--)
    {
      sh_thread_sum[tid] = 0;
      double sum = 0;
      for (int k = i + 1 + tid; k < n; k += p)
        {
          sh_thread_sum[tid] += a[n * k + i] * x[k];
        }
      synchronize (p);
      if (tid == i % p)
        {
          for (int j = 0; j < p; j++)
            {
              sum += sh_thread_sum[j];
            }
          x[i] = d[i] * (y[i] - d[i] * sum) / a[n * i + i];
        }
      synchronize (p);
    }
}
\end{verbatim}
Используется $2n$ точек синхронизации.


Здесь все потоки используют только свою память.
Сложность обычного алгоритма составляет
$$\sum_{i=1}^n\sum_{k=1}^i2 = n*(n+1)=n^2+n$$

А многопоточного, в силу того, что никакой достаточно 
большой участок кода
не выполняется с заблокированным мьютексом
$$(n^2+n)/p$$

\subsection{Оценка сложности в алгоритме построения верхнетреугольной матрицы
в параллельном блочном разложении Холецкого}

Из приведенного выше алгоритма
видно, что все потоки вычисляют совпадающие диагональные блоки
 и обратные к ним, поэтому 
 $S_0(n,m,p) = S_0(n,m)$ и $S_3(n,m,p) = S_3(n,m)$.
 Вычисление внедиагональных блоков происходит полностью параллельно:
 $S_1(n,m,p) = S_1(n,m)/p$
 \newline
 Итак, 
 $$S(n,m,p) = S_0(n,m)+S_3(n,m)+S_1(n,m)/p =$$
 $$ 
 n^3/3p- m^2n^2/6p- mn^2/3p+mn^2- n^2/2p- n^2/2 + m^3n/6p
 -m^2n/3 + mn/2p + mn/2 
 -m^3/3 
    $$
Причем $$S(n,m,1) =n^3/3-m^2n^2/6+2/3mn^2-n^2+m^3n/6-m^2n/3
+mn-m^3/3 = S(n,m)$$

\section{Описание параллельного блочного метода Холецкого с использованием
  MPI}
Отличие MPI версии от версии, которая использует потоки
заключено в том, что из одного процесса невозможно иметь доступ
ко всей матрице. Для этого будем использовать функции, которые
предоставляет стандарт MPI.
\subsection{Хранение матрицы в памяти}
Для простоты $n$ делится на $m$. Матрица представляется в виде 
$N=n/m$ блок-столбцов с треугольником, на гипотенузе которого расположены 
элементы диагонали исходной матрицы.
Процесс с рангом rank считает "своими" и выделяет для них память
блок-столбцы для которых выполнено 
$rank = I \% comm\_size$, где $I \in \{0,...,N-1\}$
\subsection{Описание алгоритма}
Алгоритм разложения
\begin{verbatim}
  Для каждой строки I из {0,.., N-1}:
      Если I % commSize == rank:
      {
          memcpy (col, columns[I], 
              (m * m * I + m * (m + 1) / 2) * sizeof (double))
      }
      MPI_Bcast (col, m * m * I + m * (m + 1) / 2, MPI_DOUBLE,
          I % commSize, MPI_COMM_WORLD)
      Triangle := col + m * m * I;
      R[I, I], D[I] := choletsky_decompose(Triangle)
      R[I, I]^(-1) := inverse(R[I, I])
      Если I % commSize == rank:
          A[I, I] = R[I, I]
      Для каждого столбца J из {i+1,.., N-1}:
          Если J % comm_size == rank:
              Для каждого блока K из {0,.., I-1}:
                  A[I, J] -= col[K]^T * D[K] * R[K, J]
              A[I, J] = (R[I, I]^-1)^T*D[I]*A[I, J]
\end{verbatim}
\subsection{Объем пересылаемых данных}
В разложении нужно $N$ раз переслать данные общим размером равным
размеру всей матрицы, то есть $n*(n+1)/2$.
При нахождении вектора $y$ нужно $N$ раз переслать данные общим размером
$n$.
При нахождении вектора $x$ нужно $N$ раз переслать данные общим размером
$n$.

В итоге количество пересылок равно $3N$, а общий объем 
пересылаемых данных $\frac{1}{2}n^2 + \frac{5}{2}n$.
\subsection{Оценка количества операций}
По сравнению с поточной версией количество операций не изменилось.
\end{document}